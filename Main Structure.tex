\documentclass[12pt, oneside]{article}
 
\usepackage{graphicx}
\usepackage{hyperref}
\graphicspath{ {Images/} }

\begin{document}
\thispagestyle{empty}
\begin{center}
\begin{minipage}{0.75\linewidth}
    \centering

    {\uppercase{\Large COS 301 Assignment\par}}
   	{\uppercase{\Large Group 2 B \par}}
    \vspace{1cm}

    {\normalsize Duran Cole (13329414)\par}
    {\normalsize Johannes Coetzee (10693077)\par}
    {\normalsize Estian Rosslee (12223426)\par}
    {\normalsize Edwin Fullard (12048675)\par}
    {\normalsize Herman Keuris (13037618)\par}
    {\normalsize Martha Mohlala (10353403)\par}
    {\normalsize Motsoape Mphahlele (12211070)\par}
    {\normalsize Xoliswa Ntshingila (13410378)\par}
    \vspace{1cm}
    
    {\Large February 2015}
\end{minipage}
\end{center}
\clearpage

\newpage

\section{Introduction}
	
\section{Architecture Requirements}
	\subsection{Architectural Scope}
	\subsection{Quality Requirements}
		\subsubsection{Scalability}
			\begin{itemize} 
				\item Manage resource demand.
				\item Scale out resources.
				\item The system must be able to operate effectively under the load of all registered students within the department of Computer Science and guest users.
			\end{itemize}
			
		\subsubsection{Performance requirements}
			\begin{itemize}
				\item Throughput: The rate at which incoming requests are completed.
				\item Manage resource demand.
				\item Minimise response time.
			
			\end{itemize}
		\subsubsection{Maintainability}
			\begin{itemize}
				\item Ability to back track to a state in which system was safe.
				\item Corrective maintenance: Reactive modification of the system performed after delivery to correct discovered problems.
				\item Adaptive maintenance: Modification of the Buzz system performed after delivery to keep the system usable in a changed or changing environment.
				\item Perfective maintenance: Modification of the Buzz system after delivery to improve performance or maintainability.
				\item Preventive maintenance: Modification of the Buzz system after delivery to detect and correct latent faults in the system before they become effective faults.
			\end{itemize}
		\subsubsection{Availability}
			\begin{itemize}
				\item Prevent faults.
				\item Detect faults.
				\item Recover from faults.
				\item Cross platform.
				\item Cross browser.
			\end{itemize}
			
		\subsubsection{Reliability}
			\begin{itemize}
				\item Prevent faults.
				\item Detect faults.
				\item Recover from faults.
			\end{itemize}
			
		\subsubsection{Security}
			\begin{itemize}
				\item Detect attacks from unwanted and unauthorised users.
				\item Resist attacks from unwanted and unauthorised users.
				\item Recover from attack from unwanted and unauthorised users.
				\item Minimize access and permissions given to users who do not have the required privileges.
				\item All communication of sensitive data must be done securely through encryption and secure channels.
				\item All system functionality is only accessible to users who can be successfully authenticated through the LDAP system used by the department of Computer Science.
			\end{itemize}
		\subsubsection{Monitorability and Auditability}
			\begin{itemize}
				\item Logs: Logs system activities such as the time a user logged into/out of the system.
							\item Each action on the system must be recorded in an audit log that can later be viewed and queried.
				\item Information to be recorded must include:
				\begin{itemize}
					\item The identity of the individual carrying out the action
					\item A description of the action
					\item When the action was carried out
				\end{itemize}
			\end{itemize}
		\subsubsection{Testability}
			\begin{itemize}
				\item Controllability: The degree to which it is possible to control the state of the component under test as required for testing by using Test driven development.
				\item Understandability: The degree to which the component under test is documented or self-explaining.
				\item Each service provided by the system must be testable through a unit test that tests:
				\begin{itemize}
					\item That the service is provided if all pre-conditions are met, and
					\item That all post-conditions hold true once the service has been provided.
				\end{itemize}
				
			\end{itemize}
		\subsubsection{Usability}
			\begin{itemize}
				\item Efficiency
				\item Ease of use
				\item Learnability
				\item Satisfaction (How pleasant it is to use the system?)
				\item The average student should be able to use the system without any prior training.
				\item Initially only English needs to be supported, but the system must allow for translations to the other official languages of the University of Pretoria to be added at a later stage.
			\end{itemize}
		\subsubsection{Integrability}
				\begin{itemize}
					\item 
				\end{itemize}
				
	\subsection{Integration and access channel requirements}
	\subsection{Architectural Constraints}		

\section{Architectural patterns or styles}
\section{Architectural tactics or strategies}
\section{Use of reference architectures and frameworks}
\section{Access and integration channels}
	\subsection{Integration Channel Used}
		\subsubsection{REST - Representational State Transfer}
		\begin{itemize}
			\item Uses standard HTTP and thus simpler to use.
		 	\item Allows different data formats where as SOAP only allows XML.
			\item Has JSON support
				\begin{itemize}
					\item faster parsing.				
				\end{itemize}			 
			\item Better performance and scalability with the ability to cache reads.
			\item Protocol Independent, can use any protocol which has a standardised URI scheme.		
		\end{itemize}
	\subsection{Protocols}
		\subsubsection{HTTP - Hypertext Transfer Protocol}
			\begin{itemize}
				\item Standard web language.
				\item Easy to write pages.
			\end{itemize}
		\subsubsection{PHP}
			\begin{itemize}
				\item Allows dynamic pages to be built.
				\item Easy integration of JavaScript and HTML with PHP functions.
			\end{itemize}
		\subsubsection{IP - Internet Protocol}	
			\begin{itemize}
				\item Allows Communications between users.
				\item In charge of sending, receiving and addressing data packets.
			\end{itemize}				
		\subsubsection{SMTP - Simple Mail Transfer Protocol}
			\begin{itemize}
				\item Sends emails.
				\item MIME (Multi-purpose Internet Mail Extensions) which allows SMTP to send multimedia files.
			\end{itemize}
		\subsubsection{TSL - Transport Layer Security}
			\begin{itemize}
				\item Alternative to SSL
				\item Newer and more secure version of SSL.
			\end{itemize}
		
\section{Technologies}
	

\end{document}