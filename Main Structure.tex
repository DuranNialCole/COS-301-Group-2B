\documentclass[12pt, oneside]{article}
 
\usepackage{graphicx}
\usepackage{hyperref}
\graphicspath{ {Images/} }

\begin{document}
\thispagestyle{empty}
\begin{center}
\begin{minipage}{0.75\linewidth}
    \centering

    {\uppercase{\Large COS 301 Assignment\par}}
   	{\uppercase{\Large Group 2 B \par}}
    \vspace{1cm}

    {\normalsize Duran Cole (13329414)\par}
    {\normalsize Johannes Coetzee (10693077)\par}
    {\normalsize Estian Rosslee (12223426)\par}
    {\normalsize Edwin Fullard (12048675)\par}
    {\normalsize Herman Keuris (13037618)\par}
    {\normalsize Martha Mohlala (10353403)\par}
    {\normalsize Motsoape Mphahlele (12211070)\par}
    {\normalsize Xoliswa Ntshingila (13410378)\par}
    \vspace{1cm}
    
    {\Large February 2015}
\end{minipage}
\end{center}
\clearpage

\newpage
\section{Introduction} This section deals with the software architecture requirements of the Buzz system being designed. It handles all aspects of the system's design which form part of its non-functional requirements, in particular the requirements of the software architecture on which the application's functional aspects are developed. This includes:
	\begin{itemize} 
		\item The architectural scope.
		\item Quality requirements.
		\item The integration and access channel requirements.
		\item The architectural constraints.
		\item Architectural patterns and styles used.
		\item The architectural tactics and strategies used.
		\item The use of reference architectures and frameworks.
		\item Access and integration channels.
		\item Technologies used.
	\end{itemize}
\newpage
\section{Architecture Requirements}
	\subsection{Architectural Scope}
	\subsection{Quality Requirements}
		\subsubsection{Scalability}
			\begin{itemize} 
				\item Manage resource demand.
				\item Scale out resources.
				\item The system must be able to operate effectively under the load of all registered students within the department of Computer Science and guest users.
			\end{itemize}
			
		\subsubsection{Performance requirements}
			\begin{itemize}
				\item Throughput: The rate at which incoming requests are completed.
				\item Manage resource demand.
				\item Minimize response time.
			\end{itemize}
		\subsubsection{Maintainability}
			\begin{itemize}
					\item Regression: Ability to backtrack to a state in which system was safe.
					\item Corrective maintenance: Reactive modification of the system performed after delivery to correct discovered problems.
					\item Adaptive maintenance: Modification of the Buzz system performed after delivery to keep the system usable in a changed or changing environment.
					\item Perfective maintenance: Modification of the Buzz system after delivery to improve performance or maintainability.
					\item Preventive maintenance: Modification of the Buzz system after delivery to detect and correct latent faults in the system before they become effective faults.
			\end{itemize}
		\subsubsection{Availability}
		\begin{itemize}
			\item Prevent faults.
			\item Detect faults.
			\item Recover from faults.
			\item Cross platform functionality.
			\item Cross browser functionality.
		\end{itemize}
		
		\subsubsection{Reliability}
		\begin{itemize}
			\item Prevent faults.
			\item Detect faults.
			\item Recover from faults.
		\end{itemize}
	
		\subsubsection{Security}
			\begin{itemize}
				\item Detect attacks from unwanted and unauthorised users.
				\item Resist attacks from unwanted and unauthorised users.
				\item Recover from attack from unwanted and unauthorised users.
				\item Minimize access and permissions given to users who do not have the required privileges.
				\item All communication of sensitive data must be done securely through encryption and secure channels.
				\item All system functionality is only accessible to users who can be successfully authenticated through the LDAP system used by the department of Computer Science.
			\end{itemize}
		\subsubsection{Monitorability and Auditability}
			\begin{itemize}
				\item Logs system activities such as the time a user logged into/out of the system
				\item Each action on the system must be recorded in an audit log that can later be viewed and queried.
				\item Information to be recorded must include:
				\begin{itemize}
					\item The identity of the individual carrying out the action.
					\item A description of the action.
					\item When the action was carried out.
				\end{itemize}
			\end{itemize}
		\subsubsection{Testability}
			\begin{itemize}
				\item Controllability: The degree to which it is possible to control the state of the component under test as required for testing.
				\item Understandability: The degree to which the component under test is documented or self-explaining.
				\item Test driven development
				\item Each service provided by the system must be testable through a unit test that tests:
				\begin{itemize}
					\item That the service is provided if all pre-conditions are met, and
					\item That all post-conditions hold true once the service has been provided.
				\end{itemize}
			\end{itemize}
		\subsubsection{Usability}
			\begin{itemize}
				\item Efficiency
				\item Ease of use
				\item Easy to navigate between the system's various web pages.
				\item Satisfaction (How pleasant is it to use the system?)
				\item The average student should be able to use the system without any prior training.
				\item Initially only English needs to be supported, but the system must allow for translations to the other official languages of the University of Pretoria to be added at a later stage.
			\end{itemize}
		\subsubsection{Integrability}
				\begin{itemize}
					\item Must be able to integrate with existing systems and systems which may want to be added.
					\item Must be able to integrate with the existing CS website.
				\end{itemize}
				
	\subsection{Integration and access channel requirements}
				\subsubsection{Human access channels}
				\begin{itemize}
					\item Must run on all major web browsers (Firefox, Internet Explorer, Opera, Chrome, Safari)
					\item Must be able to access from computer (desktop \& laptop), tablets and phones (i.e. mobile/Android devices)
					\item Must be able to able to handle thick clients (like a student's home PC) and thin clients (like the PC's in the Informatorium)
					\item Must support all major web standards such as the standards published by the International Organization for Standardization (ISO), Request for Comments (RFC), Document Object Models (DOM), proper use of HTTP, stylesheets (especially Cascading Style Sheets (CSS)) and markup languages, such as Hypertext Markup Language (HTML) and Extensible Hypertext Markup Language (XHTML).
				\end{itemize}
				\subsubsection{System access and integration channels}
				\begin{itemize}
					\item The Buzz system can be accessed through a direct web page (i.e. using http) or through a link from the CS web page.
					\item The Buzz system must be integrated seamlessly with both the CS LDAP server (to retrieve class lists and student information) and the CS MySQL database to access course and module details.
				\end{itemize}
	\subsection{Architectural Constraints}		

\section{Architectural patterns or styles}
	\subsection{Layered Architectural Style}
	\begin{itemize}
		\item The Buzz system will use ISO OSI layers and its communication protocols. For example,when accessing the Buzz space the user opens the web browser and use HTTP to directly request the Buzz space web page. This happens on the application layer.
		\item Layered architectural style supports N-tier architectural style which can improve scalability.
		\item The main benefits of layered architectural style:
			\begin{itemize}
			\item it abstracts the view of the system as whole while providing enough detail to understand the roles and responsibilities of individual layers and the relationship between them.
			\item separation of concerns which improves maintainability of the system.
			\item it improves performance and fault tolerance.
			\item it increases testability.
			\end{itemize}
		\item The Buzz system can it's simplest be implemented across the following three layers:
		\begin{enumerate}
		\item Client access layer - Provides the front-end/interface through which the client interacts with the system
		\item Business logic layer - Provides the back-end services
		\item Infrastructure layer - Includes the framework upon which the system is built and provides integration with other systems
		\end{enumerate}
	\end{itemize}
	\subsection{Virtual Machine Pattern}
	\begin{itemize}
		\item The Buzz system must be portable to run on any computing platform.That is, the system must be able to run on home desktop, PCs on campus, and different mobile devices. It must also run on different web browsers on different operating systems.
		\item Virtual machine pattern provides:
		\begin{itemize}
			\item portability
			\item pluggability
		\end{itemize}		 
	\end{itemize}
	\subsection{Client/Server Architectural Style}
	\begin{itemize}
		\item Client/Server allows multiple clients to access the system using N-tier 		architectural style.The benefit of using N-tier architectural style is that it improves the scalability of the system.
		\item The server side is the back-end of the system which manages the centralized data and access to the database from CS LDAP. The aim of having the back-end manages the data is to achieve higher security.
	\end{itemize}
	\subsection{N-tier/3-tier Architectural Style}
	\begin{itemize}
		\item The N-tier/3-tier architectural style provide improved:
		\begin{itemize}
			\item scalability
			\item availability
			\item maintainability
		\end{itemize}
		\item And resource utilization
	\end{itemize}
\section{Architectural tactics or strategies}
\section{Use of reference architectures and frameworks}
\newpage
\section{Access and integration channels}
	\subsection{Integration Channel Used}
		\subsubsection{REST - Representational State Transfer}
		\begin{itemize}
			\item Uses standard HTTP and thus simpler to use.
		 	\item Allows different data formats where as SOAP only allows XML.
			\item Has JSON support
				\begin{itemize}
					\item faster parsing.				
				\end{itemize}			 
			\item Better performance and scalability with the ability to cache reads.
			\item Protocol Independent, can use any protocol which has a standardised URI scheme.		
		\end{itemize}
	\subsection{Protocols}
		\subsubsection{HTTP - Hypertext Transfer Protocol}
			\begin{itemize}
				\item Standard web language.
				\item Easy to write pages.
			\end{itemize}
		\subsubsection{PHP}
			\begin{itemize}
				\item Allows dynamic pages to be built.
				\item Easy integration of JavaScript and HTML with PHP functions.
			\end{itemize}
		\subsubsection{IP - Internet Protocol}	
			\begin{itemize}
				\item Allows Communications between users.
				\item In charge of sending, receiving and addressing data packets.
			\end{itemize}				
		\subsubsection{SMTP - Simple Mail Transfer Protocol}
			\begin{itemize}
				\item Sends emails.
				\item MIME (Multi-purpose Internet Mail Extensions) which allows SMTP to send multimedia files.
			\end{itemize}
		\subsubsection{TSL - Transport Layer Security}
			\begin{itemize}
				\item Alternative to SSL
				\item Newer and more secure version of SSL.
			\end{itemize}
		
\section{Technologies}
	\subsection{Programming technologies}
		\begin{itemize}
			\item AJAX (Asynchronous JavaScript and XML).
			\item The JQuery library which will be used with our JavaScript.
			\item Python (used in the Django Framework).
			\item Java programming language 
		\end{itemize}
	\subsection{Web technologies}
		\begin{itemize}
			\item HTML
		\end{itemize}
	\subsection{Other technologies}
		\begin{itemize}
			\item
		\end{itemize}
\end{document}