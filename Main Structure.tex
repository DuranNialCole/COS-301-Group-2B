\documentclass[12pt, oneside]{article}
 
\usepackage{graphicx}
\usepackage{hyperref}
\graphicspath{ {Images/} }

\begin{document}
\thispagestyle{empty}

\normalfont
\begin{center}
{\fontsize{2cm}{2em}\selectfont Architecture Requirements \\}
\vspace{1cm}
\begin{minipage}{0.75\linewidth}
    \centering

    {\uppercase{\Large COS 301 Assignment\par}}
   	{\uppercase{\Large Group 2 B \par}}
    \vspace{1cm}

    {\normalsize Duran Cole (13329414)\par}
    {\normalsize Johannes Coetzee (10693077)\par}
    {\normalsize Estian Rosslee (12223426)\par}
    {\normalsize Edwin Fullard (12048675)\par}
    {\normalsize Herman Keuris (13037618)\par}
    {\normalsize Martha Mohlala (10353403)\par}
    {\normalsize Motsoape Mphahlele (12211070)\par}
    {\normalsize Xoliswa Ntshingila (13410378)\par}
    \vspace{1cm}
    
    {\Large 10 March 2015}
\end{minipage}
\end{center}
\clearpage


\pagenumbering{roman}
\tableofcontents
\newpage
\pagenumbering{arabic}

\newpage
\section{Introduction} This section deals with the software architecture requirements of the Buzz system being designed. It handles all aspects of the system's design which form part of its non-functional requirements, in particular the requirements of the software architecture on which the application's functional aspects are developed. This includes:
	\begin{itemize} 
		\item The architectural scope.
		\item Quality requirements.
		\item The integration and access channel requirements.
		\item The architectural constraints.
		\item Architectural patterns and styles used.
		\item The architectural tactics and strategies used.
		\item The use of reference architectures and frameworks.
		\item Access and integration channels.
		\item Technologies used.
	\end{itemize}
\newpage
\section{Architecture Requirements}
	\subsection{Architectural Scope}
	The following responsibilities need to be addressed by the software architecture:
	\begin{enumerate}
		\item To be able to host and provide an execution environment for the services/business logic of the Buzz system
		\item To provide an infrastructure for a web access channel
		\item To provide an infrastructure that provides a mobile access channel
		\item To provide an infrastructure that handles persisting and provides access to domain objects
		\item To integrate with a LDAP repository.
		\item To provide an infrastructure that allows module plugability in such a way that core modules are independent from add on modules and vice-versa
		\item To provide an infrastructure to integrate the system with other systems such as the CS department's website and the Hamster marking system
	\end{enumerate}
	\subsection{Quality Requirements}
		\subsubsection{Scalability}
			\begin{itemize} 
				\item Manage resource demand to ensure that every user of the Buzz system will be presented with a satisfactory experience.
				\item Scale out resources to ensure that resource demand be met.
				\item The system must be able to operate effectively under the load of all registered students within the department of Computer Science and guest users.
			\end{itemize}
			
		\subsubsection{Performance requirements}
			\begin{itemize}
				\item The Buzz system must have a large throughput number such that the rate at which incoming requests are completed is high.
				\item Minimize the response time of requests from the Buzz system to ensure that the Buzz system would work efficiently.
			\end{itemize}
		\subsubsection{Maintainability}
			\begin{itemize}
					\item Rollback to a previous state in which the Buzz system was safe. Lets us recover from an error or a system fault.
					\item Corrective maintenance is to correct discovered problems in the Buzz system after the system has been implemented.
					\item Perfective maintenance is to improve performance or maintainability after the Buzz system has been implemented.
			\end{itemize}
		\subsubsection{Availability}
		\begin{itemize}
			\item Prevent faults in the Buzz system to ensure that there is no system downtime. This will increase the availability of the Buzz system.
			\item Detect faults before the fault can cause harm to the Buzz system.
			\item Recover from faults if it could not be prevented to ensure that the Buzz system will be available once again.
			\item Cross platform functionality.
			\item Cross browser functionality.
		\end{itemize}
		
		\subsubsection{Reliability}
		\begin{itemize}
			\item Ensure that the Buzz system is reliable in such a way that all interaction with the Buzz system will give the required results.
			\item Prevent faults.
			\item Detect faults.
			\item Recover from faults.
		\end{itemize}
	
		\subsubsection{Security}
			\begin{itemize}
				\item Minimize access and permissions given to users who do not have the required privileges.
				\item All communication of sensitive data must be done securely through encryption and secure channels such that unwanted users are unable to compromise the Buzz system.
				\item All system functionality is only accessible to users who can be successfully authenticated through the LDAP system used by the department of Computer Science.
			\end{itemize}
		\subsubsection{Monitorability and Auditability}
			\begin{itemize}
				\item Each action on the system must be recorded in an audit log that can later be viewed and queried.
				\item Information to be recorded must include:
				\begin{itemize}
					\item The identity of the individual carrying out the action.
					\item A description of the action.
					\item When the action was carried out.
				\end{itemize}
			\end{itemize}
		\subsubsection{Testability}
			\begin{itemize}
				\item The Buzz system must use test driven development.
				\item Each service provided by the system must be testable through a unit test that tests:
				\begin{itemize}
					\item That the service is provided if all pre-conditions are met, and
					\item That all post-conditions hold true once the service has been provided.
				\end{itemize}
			\end{itemize}
		\subsubsection{Usability}
			\begin{itemize}
				\item The Buzz system must be efficient to use.
				\item The Buzz system must be easy to use.
				\item Easy to navigate between the system's various web pages.
				\item The average student should be able to use the system without any prior training.
				\item Initially only English needs to be supported, but the system must allow for translations to the other official languages of the University of Pretoria to be added at a later stage.
			\end{itemize}
		\subsubsection{Integrability}
				\begin{itemize}
					\item Must be able to integrate with existing systems and systems which may want to be added.
					\item Must be able to integrate with the existing CS website.
				\end{itemize}
				

	\subsection{Architectural Constraints}		
	The following architecture constraints have been imposed:
\begin{enumerate}
	\item The system must be developed using the following technologies
	\begin{itemize}
		\item Development must take place on a distribution of the Linux operating system
		\item The system must be developed using the Java-Enterprise Edition (Java-EE) development platform and the Java Server Faces (JSF) architectural framework
		\item Persistence to a relational database must be done by making use of the Java Persistence API (JPA) and the Java Persistence query language (JPQL)
		\item The unit tests should be developed using using an appropriate unit test framework such as JUnit (a unit testing framework for the Java programming language).
	\end{itemize}

	\item The system must ultimately be deployed onto a GlassFish application server running on the cs.up.ac.za Apache web server.
	\item The system must be decoupled from the choice of database. The system will use a MySQL database.
	\item The system must expose all system functionality as restful web services and hence may not have any application functionality within the presentation layer.
	\item Web services must be published as either SOAP-based or Restful web services.
\end{enumerate}
\section{Architectural patterns or styles}
	\subsection{Layered Architectural Style}
	\begin{itemize}
		\item The Buzz system can at its simplest be implemented across the following three layers:
			\begin{enumerate}
			\item Client access layer
				\begin{itemize}
				\item Provides the front-end/interface through which the client interacts with the system
				\item Thin client in the form of a web interface which will simply handle input from the user and display appropriate output
				\end{itemize}
			\item Business logic layer
				\begin{itemize}
				\item Provides the back-end services of the system
				\item Manages authentication via communication with CS LDAP
				\item Manages access to the persistence database
				\item The aim of having the back-end managing the data is to achieve higher security.
				\end{itemize}
			\item Infrastructure layer
			\begin{itemize}
				\item Includes the framework upon which the system is built and provides integration with other systems
				\end{itemize}
		\end{enumerate}		
		\item The main benefits of layered architectural style:
			\begin{itemize}
			\item it abstracts the view of the system as whole while providing enough detail to understand the roles and responsibilities of individual layers and the relationship between them.
			\item separation of concerns across different layers improves cohesion and reduces complexity
			\item improved maintainability through the ability to have separate layers developed
independently by different teams
			\item testability is improved by separation of concerns leading to smaller independent modules which can be tested
			\end{itemize}
	\end{itemize}
\section{Architectural tactics or strategies}
	\subsection{Fault Detection}
		\subsubsection{Ping / Echo}			
			\begin{itemize}		
				\item Ping / Echo refers to an asynchronous request/response message pair exchanged between nodes, used to determine reachability and the round-trip delay through the associated network path. We can use it to ensure that the students are connected and up to date with the latest version of a Buzzspace before they post something that might already have been posted.
				\item The quality requirements addressed:
					\begin{itemize}
						\item Security, as it can confirms on security policies between ports and that no faults have occurred.
						\item Availability, allows us to note whether server is available or not.
						\item Reliability, detected faults can then be acted upon. 
						\item Monitorability and Audibility, To log faults detected.
						\item Testability, allows errors to be found and dealt with.
					\end{itemize}
			\end{itemize}
			
		\subsubsection{System Monitor}
			\begin{itemize}
				\item Watchdog is a hardware-based counter-timer that is periodically reset by software. Upon expiration it indicates to the system monitor of a fault occurrence in the process.
				
				\item Heartbeat is a periodic message exchange between the system monitor and process. It indicates to system monitor when a fault is incurred in the process.
				
				\item Both of these can be used to prevent zombie threads (ones that expired or crashed on the client side) to clog up the server by taking up valuable resources.
				
				\item The quality requirements addressed:
				\begin{itemize}
					\item Availability, allows us to note whether server is available or not and view current system processes.
					\item Monitorability and Audibility, To monitor system and log activities and changes.
				\end{itemize}
			\end{itemize}	
		\subsubsection{ Exception Detection }
			\begin{itemize}
			\item System Exceptions are raised by the system when it detects a fault, such as divide by zero, bus and address faults and illegal program instructions.
			
			\item Parameter Fence incorporates an A priority data pattern (such as 0xdeadbeef) placed immediately after any variable-length parameters of an object. It allows for runtime detection of overwriting the memory allocated for the object's variable-length parameters.
			
			\item Parameter Typing employs a base class that defines functions that add, find, and iterate over Type-Length-Value (TLV) formatted message parameters and uses strong typing to build and parse messages.
			
			\item All exceptions needs to be handled in order to prevent any critical errors or failures which can lead to a system compromise.
			
			\item The quality requirements addressed:
			\begin{itemize}
				\item Security, Allows us to detect possible security errors and threats such as: buffer overflows, possible attacks, etc.
				\item Maintainability, Allows us to find and fix exceptions which could further lead to system errors and problems.
				\item Monitorability and Audibility, To monitor system and log all exceptions thrown by either system or user activities.
				\item Testability, allows exceptions to be found and dealt with.
			\end{itemize}
		\end{itemize}

	\subsection{Fault Recovery}
		
		
		\subsubsection{Preparation and Repair}
		\begin{itemize}
		
		\item Voting
		\begin{itemize}
		\item Triple Modular Redundancy is three identical processing units, each receiving identical inputs, whose output is forwarded to voting logic. It then detects any inconsistency among the three output states, which is treated as a system fault.
		\end{itemize}
		
		
		\item Active Redundancy
		\begin{itemize}
		\item Configuration wherein all of the nodes (active or redundant spare) in a protection group receive and process identical inputs in parallel. The redundant spare possesses an identical state to the active processor, so recovery and repair can occur in milliseconds.
		\end{itemize}
		
		\item Passive Redundancy
		\begin{itemize}
		\item Configuration wherein only the active members of the protection group process input traffic, with the redundant spare(s) receiving periodic state updates. This achieves a balance between the more highly available but more complex Active Redundancy tactic and the less available but significantly less complex Spare tactic.
		\end{itemize}
		
		\item Spare
		\begin{itemize}
		\item Configuration wherein the redundant spares of a protection group remain out of service until a fail-over occurs. Then it initiates the power-on-reset procedure on the redundant spare prior to its being placed in service.
		\end{itemize}
		
		\item The quality requirements addressed:
		\begin{itemize}
			\item Monitorability and Auditability, faults are detected and logged and allows us to backtrack to an earlier state or prevent faults and recover.
			\item Availability, it allows the most current operational state to be available.
			\item Maintainability, allows us to backtrack to a previous working state. 
		\end{itemize}
		
		\subsubsection{Reintroduction}
		\item Shadow
		\begin{itemize}
		\item Operates a previously failed or in-service upgraded component in a “shadow mode” for a predefined duration of time and when a problem is incurred it reverts the component back to an active role.
		\end{itemize}
		
		\item State Resynchronization 
		\begin{itemize}
		\item When it's implemented as a refinement to the Active Redundancy tactic, it occurs organically as active and standby components that each receive and process identical inputs in parallel.
		\item When it's implemented as a refinement to the Passive Redundancy tactic, it's based solely on periodic state information transmitted from the active component(s) to the standby component and involves the Rollback tactic.
		It then allows the system's control element to dynamically recover its control plane state from its network peers and periodically compares the state of the active and standby components to ensure synchronization.
		\end{itemize}
		
		\item Rollback
		\begin{itemize}
		\item Rollback is a checkpoint based tactic that allows the system state to be reverted to the most recent consistent set of checkpoints.
		\end{itemize}
		
		\item The quality requirements addressed:
		\begin{itemize}
			\item Testability, allows the system to remain intact while new components are tested.
			\item Maintainability, allows us to backtrack to a state where components where working correctly.
		\end{itemize}
		
	\end{itemize}
	
	\subsection{Fault Prevention}
		
		\subsubsection{Removal from Service}
		\begin{itemize}		
			\item Places a system component in an out-of-service state that allows for mitigating potential system failures before their accumulation affects the service.
			\item The quality requirements addressed:
			\begin{itemize}
				\item 
			\end{itemize}		
		\end{itemize}
		
		\subsubsection{Transactions}
		\begin{itemize}
			\item Ensures a consistent and durable system state. The Atomic Commit Protocol can be used and it is most commonly implemented by the two-phase commit.
			\item The quality requirements addressed:
			\begin{itemize}
				\item 
			\end{itemize}
		\end{itemize}
		
		
		\subsubsection{Process Monitor}
		\begin{itemize}
			\item Monitors the state of health of a system process and ensures that the system is operating within its nominal operating parameters.
			\item The quality requirements addressed:
			\begin{itemize}
				\item 
			\end{itemize}
		\end{itemize}



\section{Use of reference architectures and frameworks}
\newpage
\section{Access and integration channels}
	\subsubsection{Human access channels}
		\begin{itemize}
			\item Must run on all major web browsers (Mozilla Firefox, Microsoft Internet Explorer, Opera, Google Chrome, Safari)
			\item Must be able to access from any computer regardless which operating system it uses ( e.g. Windows, Linux, etc.), tablets and phones (i.e. mobile/Android devices)
			\item The user can easily use the Buzz system to communicate with both the CS LDAP server (when registering and logging on and off of their profiles) and the CS MySQL database to access course and module details.
		\end{itemize}
		\subsubsection{System access channels}
			\begin{itemize}
				\item The Buzz system can be accessed through a direct web page (i.e. using http) or through a link from the CS web page.
				\item The Buzz system must easily communicate with both the CS LDAP server (to retrieve class lists and student information) and the CS MySQL database to access course and module details.
			\end{itemize}
	\subsection{Integration Channel Used}
		\subsubsection{REST - Representational State Transfer}
		\begin{itemize}
			\item Uses standard HTTP and thus simpler to use.
		 	\item Allows different data formats where as SOAP only allows XML.
			\item Has JSON support
				\begin{itemize}
					\item faster parsing.				
				\end{itemize}			 
			\item Better performance and scalability with the ability to cache reads.
			\item Protocol Independent, can use any protocol which has a standardised URI scheme.		
		\end{itemize}
	\subsection{Protocols}
		\subsubsection{HTTP - Hypertext Transfer Protocol}
			\begin{itemize}
				\item Used to send web pages from server to workstations.  
			\end{itemize}
		\subsubsection{IP - Internet Protocol}	
			\begin{itemize}
				\item Allows Communications between users.
				\item In charge of sending, receiving and addressing data packets.
			\end{itemize}				
		\subsubsection{SMTP - Simple Mail Transfer Protocol}
			\begin{itemize}
				\item Sends emails.
				\item MIME (Multi-purpose Internet Mail Extensions) which allows SMTP to send multimedia files.
			\end{itemize}
		\subsubsection{TSL - Transport Layer Security}
			\begin{itemize}
				\item Alternative to SSL
				\item Newer and more secure version of SSL.
			\end{itemize}

		
\section{Technologies}
	\subsection{Programming technologies}
		\subsubsection{Java EE}
			\begin{itemize}
				\item Java is a platform independent language which can be run on any operating system thus allowing us to run website of a Linux machine.
				\item Contains multiple libraries that can be used that can provide functionality to the system. 
			\end{itemize}
			
		\subsubsection{JSF - Java Server Faces}
			\begin{itemize}
				\item Allows Component based user Interfaces.
				\item Gives the ability to create stateless views, page flows and portable resource contracts
			\end{itemize}
			
		\subsubsection{JPA - Java Persistence API}
			\begin{itemize}
				\item Describes how relational data is managed.
				\item Supports embedded objects.
				\item Allows shared objects to be cached.
			\end{itemize}
			
		\subsubsection{JPQL - Java Persistence Query Language}
			\begin{itemize}
				\item Structured query language is used to query the database.
				\item Allows the ability to get very specific data and preform operations on returned data by using query language.
				\item Provides the functionality of query caching.
			\end{itemize}						
		
		\subsubsection{Maven}
			\begin{itemize}
				 \item Uses Project Object Model.
				 \item Allows project documentation and reporting to be managed by central piece of data .
				 \item Provides build environment.
			\end{itemize}
			
		\subsubsection{AJAX - Asynchronous JavaScript and XML}
			\begin{itemize}
				\item JavaScript
					\begin{itemize}
						\item Provides the functionality to edit the Document Object Model(DOM).
						\item JQuery will be included as a JavaScript library, allowing the use of extended functions and methods.
					\end{itemize}
				\item XML - eXtensible Mark-up Language
					\begin{itemize}
						\item Platform independent.
						\item was created to store and transfer stored data.
						\item Easily parsing.
					\end{itemize}
			\end{itemize}
	\subsection{Web technologies}
		
		\subsubsection{HTML - Hypertext Mark-up Language/XHTML - eXtensible Hypertext Mark-up Language}	
			\begin{itemize}
				\item Standard web language.
				\item Easy to write pages.
			\end{itemize}
		\subsubsection{CSS - Cascading Style Sheet}
			\begin{itemize}
				\item Used to provide styling to HTML/XHTML web pages.
			\end{itemize}
\end{document}